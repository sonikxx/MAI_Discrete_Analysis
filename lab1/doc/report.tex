\documentclass[12pt]{article}
\usepackage[T2A]{fontenc}

\usepackage{fullpage}
\usepackage{multicol,multirow}
\usepackage{tabularx}
\usepackage{ulem}
\usepackage[utf8]{inputenc}
\usepackage[russian]{babel}
\usepackage{pgfplots}

\pgfplotsset{compat=1.9}
\setlength{\arrayrulewidth}{0.3mm}
\setlength{\tabcolsep}{12pt}
\renewcommand{\arraystretch}{1.5}

\hyphenpenalty=10000
\hbadness=10000

\newcounter{ROWNUMBERTABLE}
\newcommand\rownumber{\stepcounter{ROWNUMBERTABLE}\arabic{ROWNUMBERTABLE}}


\begin{document}


\section*{Лабораторная работа №\,1 по курсу дискрeтного анализа: сортировка за линейное время}

Выполнила студентка группы 08-208 МАИ \textit{Шевлякова София}.


\subsection*{Условие}

\begin{enumerate}
\item Требуется разработать программу, осуществляющую ввод пар «ключ-значение», их упорядочивание по возрастанию ключа указанным алгоритмом сортировки за линейное время и вывод отсортированной последовательности.
\item Вариант задания: N. 5-2 \\
Тип сортировки: поразрядная сортировка \\
Тип ключа: MD5-суммы (32-разрядные шестнадцатиричные числа) \\
Тип значения: строки переменной длины (до 2048 символов)
\end{enumerate} 


\subsection*{Метод решения}

Поразрядная сортировка имеет 2 версии, в этой лабораторной я решила использовать LSD (Least Significant Digit radix sort). При этом, сортировка элементов одного разряда может происходить любым образом, главное - чтобы эта сортировка была устойчивой, я выбрала сортировку подсчетом. Так как ключ представляет собой шестнадцатеричное число, классическую сортировку подсчетом мне пришлось доработать для обработки символов.


\subsection*{Описание программы}
Для удобного хранения элементов вида ключ-значение составлена структура данных pair, хранящая строку ключа char key[MAX\underline{ }KEY] и структуру с индексами начала и конца строки значения в непрерывном массиве, который хранит все значения элементов allData. Благодаря такому подходу, мы можем не создавать для каждого элемента массив символов на 2048 элементов, а значит оптимально используем память.\\
Основная функция программы void radix\underline{ }sort(struct pair *inputed, int index) сортирует массив inputed при помощи поразрядной сортировки, index - количество входных элементов, то есть размер массива inputed. По сути, мы несколько раз вызываем функцию сортировки подсчётом void counting\underline{ }sort(struct pair *inputed, int i, int index), которая сортирует массив inputed, по разряду i. Внутри эта функция создает новый массив res, в него она записывает отсортированный массив, а потом копирует его в исходный массив inputed.


\subsection*{Дневник отладки}

\begin{center}
\begin{tabular}{|p{0.3cm}|p{6cm} |p{6.7cm}|}
\hline
№\ & Описание ошибки &  Способ устранения \\
\hline
\rownumber & Так как я создавала статический массив структур, одним из элементов которой является массив на 2048 символов, то максимальный размер массива, который я могла задать, был 4000, что оказалось слишком мало.  & Для исправления этой ошибки я попробовала выделять память при помощи malloc, так как в этом случае память выделяется из кучи, я смогла увеличить массив до 9000. \\
\hline
\rownumber & Но массива на 9000 оказалось все равно мало для прохождения тестов. Писать свой вектор мне не очень хотелось, поэтому я продолжила искать способ увеличения размера массива.  & Для исправления я решила создать массив символов, в котором буду хранить строку, состоящую из значений всех пар, а в структуре буду теперь хранить индекс начала и конца значения для этого ключа. \\
\hline
\rownumber & Программа наконец-то прошла 11 тест, но в последующих тестах входные данные только увеличивались, что снова вызывало ошибку.  & Для исправления я просто меняла значение констант, отвечающие за размер массива, который состоит из значений каждой пары, и размер общего массива структур. \\
\hline
\end{tabular}
\setcounter{ROWNUMBERTABLE}{0}
\end{center}


\subsection*{Тест производительности}

\vspace{1em}
\def\xmin{0}
\def\xmax{1.05e6}
\def\ymin{0}
\def\ymax{1000}
\begin{tikzpicture}

\begin{axis}[
    xmin = \xmin, xmax = \xmax,
    ymin = \ymin, %ymax = \ymax,
    grid = both,
    minor tick num = 0,
    major grid style = {lightgray},
    minor grid style = {lightgray!25},
    width = 0.75\textwidth,
    height = 0.5\textwidth,
    legend pos=outer north east,
    xlabel={$N$ input data size},
    ylabel={$\tau$ execution time, \small{sec}},
    scaled ticks = false,
    xticklabel={\ifnum\ticknum=1{}\else\axisdefaultticklabel\fi},
    yticklabel={\ifnum\ticknum=1{}\else\axisdefaultticklabel\fi},
    x tick label style={
    rotate=-45,
    anchor=west
    }
    ]
    \legend{ 
	radixSort,
    };
    \addplot coordinates {
	(1e4,0.06) (3e4,0.234)
    (8e4,0.479) (1e5,0.68) 
    (3e5,1.704) (5e5,3.039)
    (7e5,4.303) (1e6,5.756)
    };
    
\end{axis}
\end{tikzpicture}
\\
Оценка сложности алгоритма: O(n $\cdot$ L), где L - число блоков (размер элемента), n - количество элементов в массиве. В нашем случае, L = 32, при постоянном L алгоритм будет линеен относительно количества входных данных.\\
Как видно из графика, рост времени работы при увеличении объема входных данных в среднем увеличивается линейно.

\subsection*{Выводы}

Выполняя лабораторную работу по курсу <<Дискретный анализ>>, я впервые работала с языком C++, научилась реализовывать поразрядную сортировку и сортировку подсчётом, вспомнила работу с памятью. Это поможет мне в ситуации, когда нужно будет написать быструю сортировку, которая будет работать за линейное время. Также я узнала, что $malloc$ выделяет память из кучи, а память под статический массив выделяется из стека.

\end{document}